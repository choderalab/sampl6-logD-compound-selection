\documentclass{article}
\usepackage[a4paper,margin=0.005in,tmargin=0.5in,landscape]{geometry}
\usepackage{booktabs}
\usepackage{longtable}
\pagenumbering{gobble}
\begin{document}
\begin{center}
\begin{longtable}{|ccccccc|}
\toprule
    ID &                                     name &                  RMSE &                   MAE &                      ME &                 R$^2$ &                     m \\
\midrule
\endhead
\midrule
\multicolumn{7}{r}{{Continued on next page}} \\
\midrule
\endfoot

\bottomrule
\endlastfoot
 hdiyq &                                    S+pKa &  0.607 [0.435, 0.748] &  0.450 [0.296, 0.608] &   0.098 [-0.127, 0.315] &  0.950 [0.914, 0.976] &  1.011 [0.934, 1.110] \\
 ccpmw &  ReSCoSS conformations // COSMOtherm pKa &  0.760 [0.588, 0.913] &  0.596 [0.434, 0.777] &  -0.221 [-0.485, 0.060] &  0.924 [0.855, 0.962] &  0.982 [0.866, 1.083] \\
 0wfzo &            Explicit solvent submission 1 &  4.889 [1.365, 7.997] &  2.170 [0.992, 3.956] &  -0.625 [-2.690, 0.948] &  0.217 [0.063, 0.808] &  0.981 [0.691, 1.280] \\
\end{longtable}
\end{center}

Notes

- Mean and 95\% confidence intervals of statistic values were calculated by bootstrapping.

- Submissions with submission IDs nb001, nb002, nb003, nb004, nb005 and nb005 include non-blind corrections to pKa predictions of only SM22 molecule.

pKas of the rest of the molecules in these submissions were blindly predicted before experimental data was released.

- pKa predictions of Epik-sequencial method (submission ID: nb007) were not blind. They were submitted after the submission deadline to be used as a reference method.

\end{document}
