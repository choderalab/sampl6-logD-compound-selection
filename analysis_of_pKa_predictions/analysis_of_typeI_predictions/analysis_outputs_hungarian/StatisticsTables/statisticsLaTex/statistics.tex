\documentclass{article}
\usepackage[a4paper,margin=0.005in,tmargin=0.5in,landscape]{geometry}
\usepackage{booktabs}
\usepackage{longtable}
\pagenumbering{gobble}
\begin{document}
\begin{center}
\begin{longtable}{|ccccccc|}
\toprule
    ID &                                     name &                  RMSE &                   MAE &                      ME &                 R$^2$ &                     m \\
\midrule
\endhead
\midrule
\multicolumn{7}{r}{{Continued on next page}} \\
\midrule
\endfoot

\bottomrule
\endlastfoot
 hdiyq &                                    S+pKa &  0.624 [0.478, 0.761] &  0.468 [0.328, 0.623] &   0.127 [-0.095, 0.340] &  0.950 [0.920, 0.974] &  0.990 [0.918, 1.096] \\
 ccpmw &  ReSCoSS conformations // COSMOtherm pKa &  0.788 [0.623, 0.931] &  0.622 [0.449, 0.790] &  -0.169 [-0.438, 0.100] &  0.919 [0.862, 0.956] &  0.948 [0.819, 1.058] \\
 0wfzo &            Explicit solvent submission 1 &  2.894 [1.713, 3.998] &  1.880 [1.162, 2.769] &   0.762 [-0.175, 1.835] &  0.479 [0.202, 0.761] &  0.995 [0.586, 1.371] \\
\end{longtable}
\end{center}

Notes

- Mean and 95\% confidence intervals of statistic values were calculated by bootstrapping.

- Submissions with submission IDs nb001, nb002, nb003, nb004, nb005 and nb005 include non-blind corrections to pKa predictions of only SM22 molecule.

pKas of the rest of the molecules in these submissions were blindly predicted before experimental data was released.

- pKa predictions of Epik-sequencial method (submission ID: nb007) were not blind. They were submitted after the submission deadline to be used as a reference method.

\end{document}
